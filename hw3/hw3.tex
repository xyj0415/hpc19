\documentclass[12pt]{article}
\usepackage[utf8]{inputenc}
\usepackage{url}

\title{High Performance Computing Assignment 2}
\author{Yijian Xie}
\date{March 31, 2019}

\begin{document}

\maketitle
The processor I use is i5-7267U @ 3.10GHz. It has 2 processors and 4 threads.

\section{Approximating Special Functions Using Taylor Series \& Vectorization}
    I implemented the AVX part of function \verb|sin4_intrin()|, and the result is shown below.

    Reference time: 0.2998

    Taylor time:    1.7684      Error: 6.927903e-12

    Intrin time:    0.0027      Error: 6.927903e-12

    The function \verb|sin4_intrin()| is about 100x faster than the \verb|sin()| function.

    I also found there is an intrinsic function(\verb|_mm256_sin_pd()|) which can calculate sin($x$) directly, and a group of intrinsic functions that use 512-bit registers. However the architecture of my CPU does not support any of them, so I left these code (\verb|sin4_intrin2()|) commented out.

\section{Parallel Scan in OpenMP}
    I modified the \verb|scan_seq()| function so that $prefix\_sum[i] = \sum_{k=0}^{i}A[i]$ to match the description in the problem.

    Thread=4:

    sequential-scan = 0.434113s

    parallel-scan   = 0.233390s

    Thread=2:

    sequential-scan = 0.416888s
    
    parallel-scan   = 0.293155s
\end{document}